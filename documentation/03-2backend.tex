\subsection{Logika aplikacji (ang. \textit{backend})}

\subsubsection{Punkty dostępowe (ang. \textit{enpoint})}
Dzięki ogromnej popularności aplikacji internetowych opartych na języku \textit{Java} oraz \textit{framework'a Spring} możliwe było szybkie wygenerowanie dokumentacji \textit{Open API}. Pod \href{https://trunk-kartapacjentaservice.herokuapp.com/swagger-ui.html} {linkiem} dostępny jest spis wszystkich dostępnych w serwisie endpointów. Wejście w ten link będzie wymagało podania loginu i hasła (dostępnego tutaj: \ref{credentials}).

\clearpage
\input{03-01restapi.tex}

\subsubsection{Zabezpieczenie danych - API}
Większość punktów dostępowych dostępnych w serwisie zabezpieczone jest przy użyciu metody \textit{Basic Auth}. Bez podania loginu i hasła niemożliwy jest dostęp do serwisu. Jedyne dostępne bez konieczności autoryzacji punkty dostępowe to te dotyczące logowania i rejestracji.

\subsubsection{Zabezpieczenie danych - baza danych}
Do zabezpieczenia danych skorzystaliśmy z symetrycznego szyfrowania. Informacje przechowywane w bazie są niemożliwe do odszyfrowania bez użycia klucza. Dane w punktach dostępowych są odszyfrowane. Odszyfrowywaniem zajmuje się aplikacja odpowiadająca za logikę serwisu.

\begin{figure}[H]
\centering
\includegraphics[width=15cm]{pictures/bd-encr}
\caption{Zaszyfrowane krotki bazy danych. Widoczne jest, że ciągi znaków zapisane w bazie danych nie są możliwe do odszyfrowania bez dekodowania.}
\end{figure}

\todo{Tu skończyłem}
\subsubsection{Ograniczenia}
W trakcie implementacji kolejnych funkcjonalności musieliśmy zmierzyć się z ograniczeniami serwera Heroku. Obsługa requestów jest jednowątkowa, stąd testowanie jego wydajności jest mocno ograniczone.
Skorzystanie z darmowej domeny sprawia, że funkcjonowanie strony jest po prostu wolne.

\subsubsection{Możliwości dotyczące rozwoju - przyspieszenie}
Przyspieszenie może zostać uzyskane np. poprzez zastosowanie architektury mikroserwisowej, tak by każda atomowa operacja mogła zostać wykonywana niezaleźnie. Zapewniłoby to dużą skalowalność systemU i pod dużym obciązeniem przełożyłoby się to na przyspieszenie. 

\todo

