\subsection{Przeznaczenie aplikacji}
Aplikacja ma służyć do zamawiania przez nas artykułów spożywczych z dowolnego sklepu. Osoba,
która będzie korzystała z aplikacji będzie mogła wybrać produkt np. chleb, następnie wybiera z jakiego
sklepu ten produkt ma być przywieziony. Oczywiście będzie miała wybór ze sklepów, które będą
dostępne aktualnie w aplikacji. Po złożeniu zamówienia, użytkownik otrzyma czas w jakim zamówienie
zostanie do niego dostarczone. Aplikacja działa na zasadzie zamówienie taksówki, tylko że zamawiamy
jedzenie.

\subsection{Opis Interfejsu}
Podczas otwierania aplikacji wyświetli się logo a następnie zostaniemy przeniesieni na stronę, w której
przy pierwszym rozpoczęciu będziemy musieli się zalogować i podłączyć kartę płatniczą. Na głównej
stronie trzeba będzie podać adres dostawy, chyba że pozwoliliśmy aplikacji na korzystanie z lokalizacji
telefonu. Następnie wybieramy jakie produkty chcemy zamówić, będzie wyświetlało się zdjęcie
produktu, cena, liczba oraz z jakiego sklepu zostanie przywieziony. Po dokonaniu wyboru otrzymamy
informacje w jakim czasie otrzymamy dostawę. Aplikacja w sidenavie będzie miała następujące pola:
Zamówienie, Płatności, historia zamówień, śledzenie aktualnej przesyłki, jeżeli dostawca będzie miał
opcje śledzenia , pomoc oraz informacje odnośnie aplikacji. W płatnościach możemy zmienić naszą
kartę płatniczą. W historii zamówienia będziemy mogli prześledzić nasze wszystkie zamówienia.

\subsection{Oprawa graficzna}
Szata graficzna w aplikacji będzie stała. Kolory, która będą przeważały to granatowy niebieski
połączony z białym. Czcionka będzie dosyć duża z racji tego, że aplikacja ma prostą budową i nie
posiada zbyt dużo funkcji.