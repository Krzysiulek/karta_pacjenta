\section{Opis projektu ,,Karta Pacjenta''}
Projekt ,,Karta Pacjenta'' zakładał stworzenie aplikacji umożliwiającej przechowywanie danych pacjentów w serwisie bazodanowym.
Aplikacja przechowuje informacje dotyczące chorób przebytych przez pacjenta, wystawionych przez lekarza recept.
Umożliwia bezpieczne przechowywanie danych wrażliwych, takich jak pesel, numer telefonu itd.
Umożliwia także prezentację tych danych oraz ich eksport (także w postaci anonimowej - bez danych osobowych -
posiadających tylko informację o przebytej chorobie, nie o pacjencie).

Aplikacja powstała przy użyciu języków: Java (ang. \textit{backend}) oraz Angular (TypeScript) (ang. \textit{frontend}).
Serwis postawiony jest na darmowej domenie dostępnej pod \href{http://trunk-kartapacjenta.herokuapp.com/}{tym linkiem}.
Gorąco zachęcamy do zapoznania się z działaniem aplikacji.

Dzięki serwisowi \href{https://heroku.com/} {Heroku} możliwe było darmowe opublikowanie witryny\\ w internecie.
Nawet w darmowej wersji serwis ten zapewnia usługi związane z CI (ang. \textit{Continuous Integration}).
Efektem tego, jest fakt, że po każdej aktualizacji zdalnego repozytorium \textit{Git} serwis automatycznie przebudowuje się.

\subsection{Przedstawienie aplikacji}

% login page
\begin{figure}[H]
\centering
\includegraphics[width=15cm]{pictures/service/01_login}
\caption{Ekran logowania.}
\end{figure}

% admin panel
\begin{figure}[H]
\centering
\includegraphics[width=15cm]{pictures/service/02-admin_panel}
\caption{Panel administratora, umożliwiający nadawanie praw, oraz usuwanie użytkowników.}
\end{figure}

% choroby
\begin{figure}[H]
\centering
\includegraphics[width=15cm]{pictures/service/04-choroby}
\caption{Zakładka zawierająca wypisane wszystkie zapisane w systemie choroby.
Obok widoczna zakładka umożliwiająca dodanie nowej choroby. Przycisk \textit{,,info''} w tabeli \textit{,,More info''}
umożliwia zapoznanie się z dostępnymi informacjami dotyczącymi danej choroby.}
\end{figure}

% patients
\begin{figure}[H]
\centering
\includegraphics[width=15cm]{pictures/service/03-patients}
\caption{Zakładka ze wszystkimi dostępnymi pacjentami.}
\end{figure}

%patient-ingo
\begin{figure}[H]
\centering
\includegraphics[width=15cm]{pictures/service/05-patient_info}
\caption{Zakładka zawierająca informacje o wybranym pacjencie.}
\end{figure}

% choroby pacjenta
\begin{figure}[H]
\centering
\includegraphics[width=15cm]{pictures/service/06-choroby_pacjenta}
\caption{Historia wizyt pacjenta.}
\end{figure}

% historia normalna
\begin{figure}[H]
\centering
\includegraphics[width=15cm]{pictures/service/09-history_normal}
\caption{Szczegółowa historia wizyt}
\end{figure}

% json normal
Serwis umożliwia wygenerowanie danych o pacjencie i przebytych chorobach w zunifikowanym formacie \textit{JSON}. 
\begin{figure}[H]
\centering
\includegraphics[width=15cm]{pictures/service/07-json_normal}
\caption{Dane pacjenta i jego choroby w formacie JSON.}
\end{figure}

% swagger
Serwis posiada także możliwość przetestowania punktów końcowych (ang. \textit{endpoint}). Funkcjonalność dostępna jest na  \href{https://trunk-kartapacjentaservice.herokuapp.com/swagger-ui.html#/}{stronie}.

\subsection{Testowanie aplikacji}
Do przetestowania wszystkich funkcji aplikacji ,,Karta Pacjenta'' wymagane jest, aby skorzystać z konta administratora - konta różnych użytkowników posiadają różne uprawnienia. Administrator ma dostęp do wszystkich możliwych miejsc na stronie. Podczas testowania aplikacji proszę używać konta:\\
\begin{center}
\label{credentials}
login: admin\\
hasło: admin
\end{center}