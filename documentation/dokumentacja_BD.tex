\documentclass[12pt,a4paper]{article}
\usepackage[polish]{babel}
\usepackage[T1]{fontenc}
\usepackage[utf8x]{inputenc}
\usepackage{hyperref}
\usepackage{url}
\usepackage{graphicx}
\usepackage{float}
\usepackage{xcolor,listings}
\usepackage{textcomp}
\lstset{upquote=true}

\usepackage{color}
%\newcommand{\todo}[1]{\\ \textcolor{red}{TODO: #1}\PackageWarning{TODO:}{#1!}}
\newcommand{\todo}[1]{\\ \textcolor{red}{TODO: #1}}

\addtolength{\hoffset}{-1.5cm}
\addtolength{\marginparwidth}{-1.5cm}
\addtolength{\textwidth}{3cm}
\addtolength{\voffset}{-1cm}
\addtolength{\textheight}{2.5cm}
\setlength{\topmargin}{0cm}
\setlength{\headheight}{0cm}

\begin{document}

\title{Bazy Danych\\Projekt Karta Pacjenta}
\date{\today}

\maketitle

\tableofcontents
\clearpage

% zad 1
\section{Autorzy}
   \large
{
\textbf{Krzysztof Czarnecki}  - implementacja niewidocznej dla użytkownika części programu, która posiada dostęp do wymaganych zasobów (ang. \textit{backend}) \\
\textbf{Błażej Czekała} - implementacja testów obciążeniowych, z użyciem wielowątkowości\\
\textbf{Patryk Wenz} - implementacja wyglądu i zachowania strony (ang. \textit{frontend})\\
\textbf{Hubert Braun} - implementacja bazy danych i jej obsługi
} 
\section{Opis projektu ,,Karta Pacjenta''}
Projekt ,,Karta Pacjenta'' zakładał stworzenie aplikacji umożliwiającej przechowywanie danych pacjentów w serwisie bazodanowym.
Aplikacja przechowuje informacje dotyczące chorób przebytych przez pacjenta, wystawionych przez lekarza recept.
Umożliwia bezpieczne przechowywanie danych wrażliwych, takich jak pesel, numer telefonu itd.
Umożliwia także prezentację tych danych oraz ich eksport (także w postaci anonimowej - bez danych osobowych -
posiadających tylko informację o przebytej chorobie, nie o pacjencie).

Aplikacja powstała przy użyciu języków: Java (ang. \textit{backend}) oraz Angular (TypeScript) (ang. \textit{frontend}).
Serwis postawiony jest na darmowej domenie dostępnej pod \href{http://trunk-kartapacjenta.herokuapp.com/}{tym linkiem}.
Gorąco zachęcamy do zapoznania się z działaniem aplikacji.

Dzięki serwisowi \href{https://heroku.com/} {Heroku} możliwe było darmowe opublikowanie witryny\\ w internecie.
Nawet w darmowej wersji serwis ten zapewnia usługi związane z CI (ang. \textit{Continuous Integration}).
Efektem tego, jest fakt, że po każdej aktualizacji zdalnego repozytorium \textit{Git} serwis automatycznie przebudowuje się.

\subsection{Przedstawienie aplikacji}

% login page
\begin{figure}[H]
\centering
\includegraphics[width=15cm]{pictures/service/01_login}
\caption{Ekran logowania.}
\end{figure}

% admin panel
\begin{figure}[H]
\centering
\includegraphics[width=15cm]{pictures/service/02-admin_panel}
\caption{Panel administratora, umożliwiający nadawanie praw, oraz usuwanie użytkowników.}
\end{figure}

% choroby
\begin{figure}[H]
\centering
\includegraphics[width=15cm]{pictures/service/04-choroby}
\caption{Zakładka zawierająca wypisane wszystkie zapisane w systemie choroby.
Obok widoczna zakładka umożliwiająca dodanie nowej choroby. Przycisk \textit{,,info''} w tabeli \textit{,,More info''}
umożliwia zapoznanie się z dostępnymi informacjami dotyczącymi danej choroby.}
\end{figure}

% patients
\begin{figure}[H]
\centering
\includegraphics[width=15cm]{pictures/service/03-patients}
\caption{Zakładka ze wszystkimi dostępnymi pacjentami.}
\end{figure}

%patient-ingo
\begin{figure}[H]
\centering
\includegraphics[width=15cm]{pictures/service/05-patient_info}
\caption{Zakładka zawierająca informacje o wybranym pacjencie.}
\end{figure}

% choroby pacjenta
\begin{figure}[H]
\centering
\includegraphics[width=15cm]{pictures/service/06-choroby_pacjenta}
\caption{Historia wizyt pacjenta.}
\end{figure}

% historia normalna
\begin{figure}[H]
\centering
\includegraphics[width=15cm]{pictures/service/09-history_normal}
\caption{Szczegółowa historia wizyt}
\end{figure}

% json normal
Serwis umożliwia wygenerowanie danych o pacjencie i przebytych chorobach w zunifikowanym formacie \textit{JSON}. 
\begin{figure}[H]
\centering
\includegraphics[width=15cm]{pictures/service/07-json_normal}
\caption{Dane pacjenta i jego choroby w formacie JSON.}
\end{figure}

% swagger
Serwis posiada także możliwość przetestowania punktów końcowych (ang. \textit{endpoint}). Funkcjonalność dostępna jest na  \href{https://trunk-kartapacjentaservice.herokuapp.com/swagger-ui.html#/}{stronie}.

\subsection{Testowanie aplikacji}
Do przetestowania wszystkich funkcji aplikacji ,,Karta Pacjenta'' wymagane jest, aby skorzystać z konta administratora - konta różnych użytkowników posiadają różne uprawnienia. Administrator ma dostęp do wszystkich możliwych miejsc na stronie. Podczas testowania aplikacji proszę używać konta:\\
\begin{center}
\label{credentials}
login: admin\\
hasło: admin
\end{center}
\section{Wykorzystane technologie}
\begin{itemize}
	\item Język Programowania \textbf{Java} - (ang. \textit{backend}),
	\item Szkielet aplikacyjny (ang. \textit{framework}) \textbf{Spring Boot} - wykorzystywany do programowania logiki serwisu internetowego (ang. \textit{backend}),
	\item Szkielet aplikacyjny (ang. \textit{framework}) \textbf{Spring Security} - wykorzystywany do zabezpieczenia dostępów do punktów dostępowych serwisu internetowego (ang. \textit{endpoint}),
	\item Szkielet aplikacyjny (ang. \textit{framework}) \textbf{Angular CLI} - wykorzystywany do utworzenia warstwy wizualnej aplikacji (ang. \textit{frontend}),
	\item System zarządzania relacyjnymi bazami danych \textbf{PostgreSQL}.
\end{itemize}

\section{Implementacja}
\subsection{Kontrola wersji}
Do pracy zespołowej wykorzystano narzędzie \textit{Git}. Umożliwiło ono sprawne dzielenie się zmianami w kodzie, zarządzanie i wersjonowanie zmian.

\input{03-01baza.tex}
\subsection{Logika aplikacji (ang. \textit{backend})}

\subsubsection{Punkty dostępowe (ang. \textit{enpoint})}
Dzięki ogromnej popularności aplikacji internetowych opartych na języku \textit{Java} oraz \textit{framework'a Spring} możliwe było szybkie wygenerowanie dokumentacji \textit{Open API}. Pod \href{https://trunk-kartapacjentaservice.herokuapp.com/swagger-ui.html} {linkiem} dostępny jest spis wszystkich dostępnych w serwisie endpointów. Wejście w ten link będzie wymagało podania loginu i hasła (dostępnego tutaj: \ref{credentials}).

\subsubsection{Dlaczego REST?}
Zalety REST API:
\begin{itemize}
    \item Bezstanowość klienta - serwer nie ma potrzeby zapamiętywania wcześniejszego stanu, ponieważ zapytania HTTP zawierają wszystkie potrzebne informacje,
    \item Łatwość manipulowania obiektami z poziomu URL - metodami HTTP,
    \item Czytelność wykonywanych działań ze względu na używanie metod HTTP zgodnych z ich przeznaczeniem (np, DELETE do usunięcia danych, GET do pobrania),
    \item Uniwersalność odpowiedzi serwisu - możliwe jest użycie tych samych danych wygenerowanych przez serwis do obsługi aplikacji klienckich na różnych urządzeniach (np. przeglądarka i aplikacja mobilna).
\end{itemize}

\subsubsection{Zabezpieczenie danych - API}
Większość punktów dostępowych dostępnych w serwisie zabezpieczone jest przy użyciu metody \textit{Basic Auth}. Bez podania loginu i hasła niemożliwy jest dostęp do serwisu. Jedyne dostępne bez konieczności autoryzacji punkty dostępowe to te dotyczące logowania i rejestracji.

\subsubsection{Zabezpieczenie danych - baza danych}
Do zabezpieczenia danych skorzystaliśmy z symetrycznego szyfrowania. Informacje przechowywane w bazie są niemożliwe do odszyfrowania bez użycia klucza. Dane w punktach dostępowych są odszyfrowane. Odszyfrowywaniem zajmuje się aplikacja odpowiadająca za logikę serwisu.

\begin{figure}[H]
\centering
\includegraphics[width=15cm]{pictures/bd-encr}
\caption{Zaszyfrowane krotki bazy danych. Widoczne jest, że ciągi znaków zapisane w bazie danych nie są możliwe do odszyfrowania bez dekodowania.}
\end{figure}

\subsubsection{Ograniczenia}
W trakcie implementacji kolejnych funkcjonalności musieliśmy zmierzyć się z ograniczeniami serwera Heroku. Obsługa zapytań (ang. \textit{request}) jest wykonywana na ograniczonym serwerze, stąd można zaobserwować wydłużony czas oczekiwania na duże zapytanie. Skorzystanie z darmowej domeny sprawia, że funkcjonowanie strony jest po prostu wolne.

\subsubsection{Przyspieszenie}
W trakcie pierwszej wersji implementacji zastosowano wbudowany we framework Spring sterownik będący mostkiem pomiędzy obiektami programu a bazą danych. Jego zastosowanie umożliwiło stosowanie zapytań do bazy przy użyciu specjalnie spreparowanych nazw metod. Wykorzystując ten sposób i np. metodę

\begin{lstlisting}
Optional<Patient> findByUserId(Long userId);
\end{lstlisting}

automatycznie generuje się kod SQL, który odpowiada za znalezienie użytkownika o zadanym ID.

Jest to bardzo wygodne, jednak korzystając z możliwości zbudowania własnego zapytania przy użyciu języka SQL udało się uzyskać \textbf{czterokrotne} przyspieszenie związane ze stosowaniem bardziej złożonych zapytań. Wstrzyknięcie zapytania SQL zaimplementowane jest w następujący sposób:

\begin{lstlisting}
@Query(
value = "select distinct patients.patient_id, 
	my_app_users.* from my_app_users " +
	"join patients\n" +
	"on my_app_users.user_id=patients.user_id",
	nativeQuery = true)
List<PatientInfoTO> findAllPatients();
\end{lstlisting}



\subsubsection{Możliwości dotyczące rozwoju - przyspieszenie}
Przyspieszenie może zostać uzyskane np. poprzez zastosowanie architektury mikro-serwisowej, tak by każda atomowa operacja mogła zostać wykonywana niezależnie. Zapewniłoby to dużą skalowalność systemU i pod dużym obciążeniem przełożyłoby się to na przyspieszenie. 


\subsection{Warstwa wizualna aplikacji ,,Karta Pacjenta'' (ang. \textit{frontend})}
\subsubsection{Działanie warstwy wizualnej}
Warstwa wizualna oparta jest na koncepcie reakcyjnego obsługiwania zdarzeń użytkownika końcowego. Zamiast przeładowania strony po wykonaniu akcji związanej z zdarzeniem wejścia aplikacji widoku preferowane jest odświeżenie jej części.

Frontend i Backend zostały oddzielone od siebie podczas implementacji - osobne repozytoria git.
W części Frontendu skorzystano z Bootsrapa, a obsługę akcji wykonuje inny framework TypeScryptowy - Angular.
Rozdzielnie projektu w taki sposób, umożliwia oddzielną pracę nad tymi częściami.
Frontend jest kanałem komunikacji między serwerem a klientem. Taka zależność zapewnia szybkie wykonywanie akcji
zadanych przez użytkownika bez znajomości wewnętrznej implementacji serwisu.
Podczas tworzenia "Karty Pacjenta" starano się, aby wystrój był maksymalnie przejrzysty,
wszystkie funkcjonalności opisane i rozmieszczone w jednoznaczny sposób, a komunikacja między serwerem a klientem była jak najszybsza.

\subsection{Testy obciążeniowe}

\subsubsection{Testowane zagadnienia}
Z uwagi na to, że dostęp do aplikacji powinno mieć w tym samym czasie dziesiątki tysięcy użytkowników (pacjenci i lekarze jednocześnie).
Stąd bardzo ważną rzeczą jest obsługa endpointów, które dostarczają odpowieni zbiór danych.\\
Zaimplementowane zostały testy obciążeniowe mające na celu sprawdzenie wydajności naszego serwisu.
Skorzystaliśmy z RestTemplate, czyli biblioteki dostępnej we frameworku Spring.\\
Badane zagadnienia:
\begin{enumerate}
\item Obsługa requestów http, metod GET, POST, PUT, DELETE,
\item Weryfikacja statusów HTTP,
\item Czas odpowiedzi serwera,
\item Odporność na żądania wielu (dziesiątki tysięcy) użytkowników.
\end{enumerate}

\subsubsection{Wielowątkowość}
\subsubsection{Wielowątkowość}
Wielowątkowość została zaimplementowana po to, aby możliwie najlepiej oddać realny sposób użytkowania.
W jednym momencie tysiące użytkowników może zażądać tego samego zbioru danych. Każde z urządzeń powinno zostać poprawnie obsłużone.
\subsubsection{Szybkość działania aplikacji}
\subsubsection{Szybkość działania aplikacji}
Podczas przeprowadzanych testów wydajnościowych zaobserwowano spadek wydajności działania serwisu. Podczas próby pobrania przez jednego użytkownika listy wszystkich dostępnych pacjentów czas wykonania zapytania wynosił 350ms. Podczas symulowanych testów na 1000 użytkownikach próbujących otrzymać dostęp do serwisu zmierzono średni czas dostępu do pojedynczego punktu dostępowego ok. 800ms. 
\subsubsection{Rola testów w rozwoju aplikacji}
Testy pomagają zweryfikować, czy postawione endpointy spełniają swoją rolę.



\nocite{craigspring}
\nocite{sql}
\nocite{algorytmy}
\nocite{springporadnik}
\nocite{springsecurity}
\nocite{rest}
\nocite{angular}

\bibliographystyle{unsrtnat}
\bibliography{references}

\end{document}

% PUNKTORY
% \begin{itemize}
% \item Public Documents
% \item All Documents
% \item Create Document
% \end{itemize}

% obrazek
% \begin{figure}[H]
% \centering
% \includegraphics[width=15cm]{pictures/deszyf_mycbc.png}
% \caption{Wykres zależności czasu deszyfrowania [ms] od wielkości pliku [MB] z uwzględnieniem własnej implementacji szyfru CBC}
% \label{pictures/szyfrowanie.png}
% \end{figure}