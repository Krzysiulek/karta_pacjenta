\subsubsection{Testowane zagadnienia}
Z uwagi na to, że dostęp do rzeczywistego systemu powinno mieć w tym samym czasie setki użytkowników (pacjenci i lekarze jednocześnie) ważną rolą w tworzeniu aplikacji jest jej testowanie. Oprócz testów jednostkowych, które weryfikują poprawność implementacji zdecydowano się również zaimplementować testy obciążeniowe, symulujące rzeczywiste zachowania użytkowników na systemie. 

Testy obciążeniowe wykonane zostały z wykorzystaniem wielowątkowości. Pozwoliła ona symulować sytuację, w której do jednego punktu dostępowego serwisu w jednym czasie próbuje uzyskać dostęp wielu użytkowników. 

Implementacja testów mająca na celu weryfikację wydajności naszego serwisu wykonana została przy użyciu biblioteki \textit{Rest Template} w języku Java.

Badane zagadnienia:
\begin{enumerate}
\item Obsługa requestów http, metod GET, POST, PUT, DELETE,
\item Weryfikacja statusów HTTP,
\item Czas odpowiedzi serwera,
\item Odporność na żądania wielu (dziesiątki tysięcy) użytkowników.
\end{enumerate}
