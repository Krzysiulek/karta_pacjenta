\subsubsection{Endpointy}
Dzięki ogromnej popularności aplikacji internetowych opartych na języku \textit{Java} oraz \textit{framework Spring} możliwe było szybkie wygenerowanie dokumentacji \textit{Open API}. Pod \href{https://trunk-kartapacjentaservice.herokuapp.com/swagger-ui.html} {linkiem} dostępny jest spis wszystkich dostępnych w serwisie endpointów. Wejście w ten link będzie wymagało podania loginu i hasła (dostępnego tutaj: \ref{credentials}).

\subsubsection{Dlaczego REST?}

\subsubsection{Zabezpieczenie danych - API}
Większość \textit{endpointów} dostępnych w serwisie zapezpieczone jest przy użyciu metody \textit{Basic Auth}. Bez podania loginu i hasła niemożliwy jest dostęp do serwisu. Jedyne dostępne bez konieczności autoryzacji endpointy to te dotyczące logowania i rejestracji.

\subsubsection{Zabezpieczenie danych - baza danych}


\subsubsection{Ograniczenia}
\subsubsection{Możliwości dotyczące rozwoju - przyspieszenie}
\subsubsection{Rola serwera CRUD w aplikacji}